\documentclass[10pt, compress]{beamer}

\usetheme[numbering=fraction, progressbar=none, titleformat=smallcaps, sectionpage=none]{metropolis}

\usepackage{sourcecodepro}
\usepackage{booktabs}
\usepackage{array}
\usepackage{listings}
\usepackage{graphicx}
\usepackage[english]{babel}
\usepackage[scale=2]{ccicons}
\usepackage{url}
\usepackage{relsize}
\usepackage{wasysym}
\usepackage{animate}

\usepackage{pgfplots}
\usepgfplotslibrary{dateplot}

\definecolor{Base}{HTML}{191F26}
\definecolor{Accent}{HTML}{157FFF}

\setbeamercolor{alerted text}{fg=Accent}
\setbeamercolor{frametitle}{bg=Base}

\setsansfont[BoldFont={Source Sans Pro Semibold},
              Numbers={OldStyle}]{Source Sans Pro}

\lstset{ %
  backgroundcolor={},
  basicstyle=\ttfamily\footnotesize,
  breakatwhitespace=true,
  breaklines=true,
  captionpos=n,
  commentstyle=\color{orange},
  escapeinside={\%*}{*)},
  extendedchars=true,
  frame=n,
  keywordstyle=\color{Accent},
  language=C++,
  rulecolor=\color{black},
  showspaces=false,
  showstringspaces=false,
  showtabs=false,
  stepnumber=2,
  stringstyle=\color{gray},
  tabsize=2,
  keywords={thrust,plus,device_vector, copy,transform,begin,end, copyin,
  copyout, acc, \_\_global\_\_, void, int, float, main, threadIdx, blockIdx,
  blockDim, if, else, malloc, NULL, cudaMalloc, cudaMemcpy, cudaSuccess,
  cudaGetLastError, cudaDeviceSynchronize, cudaFree, cudaMemcpyDeviceToHost,
  cudaMemcpyHostToDevice, const, data, independent, kernels, loop,
  fprintf, stderr, cudaGetErrorString, EXIT_FAILURE, for, dim3, pthread_t,
  pthread_create, exit, pthread_exit, long, printf},
  otherkeywords={::, \#pragma, \#include, \#define, <<<,>>>, \&, \*, +, -, /, [, ], >, <}
}

\renewcommand*{\UrlFont}{\ttfamily\smaller\relax}

\graphicspath{{../img/}}

\title{POSIX Threads}
\author{\footnotesize Pedro Bruel \\ {\scriptsize \emph{phrb@ime.usp.br}}}
\institute{\includegraphics[height=2cm]{imelogo}\\[0.2cm] Instituto de Matemática e Estatística \\ Universidade de São Paulo}
\date{\scriptsize \today}

\begin{document}

\maketitle

\section*{Introdução}

\subsection*{Roteiro}

\begin{frame}
    \frametitle{Roteiro}
    \setbeamertemplate{section in toc}[sections numbered]
    \tableofcontents[hideallsubsections]
\end{frame}

\begin{frame}
    \frametitle{Slides}
    \begin{center}
        \includegraphics[width=.18\textwidth]{github}
    \end{center}
    The slides and all source code are hosted at \alert{GitHub}:

    \begin{itemize}
        \item \url{github.com/phrb/aula-pthreads}
    \end{itemize}
\end{frame}

\section{Motivação}

\begin{frame}
    \frametitle{Programação Concorrente: Motivação}
    \alert{Desempenho}:
    \begin{itemize}
        \item Arquiteturas paralelas
        \item Memória Compartilhada
        \item SMP, hyperthreaded, multi-core, NUMA, $\dots$
    \end{itemize}

    \alert{Modelagem}:
    \begin{itemize}
        \item Descrever paralelismo natural
        \item Tarefas independentes
    \end{itemize}
\end{frame}

\begin{frame}
    \frametitle{Programação Concorrente: Motivação}
    \begin{center}
        \includegraphics[width=.55\textwidth]{shared_work}
    \end{center}
\end{frame}

\begin{frame}
    \frametitle{Programação Concorrente: Motivação}
    \begin{center}
        \animategraphics[loop,autoplay,width=\linewidth]{12}{puppies/puppies-}{0}{68}
    \end{center}
\end{frame}

\section{Processos \& Threads}

\begin{frame}
    \frametitle{Processos \& Threads}
    Cada \alert{processo} tem seu próprio \alert{espaço de endereçamento}:
        \begin{itemize}
            \item \alert{Segmento de texto}: código executável
            \item \alert{Segmento de dados}: variáveis globais
            \item \alert{Pilha}: dados temporários
        \end{itemize}

    \alert{Contexto} do processo:
        \begin{itemize}
            \item Conjunto mínimo de dados
            \item Contador do programa
            \item Registradores
            \item Gravado quando o processo é interrompido
            \item Relido quando é retomado
        \end{itemize}
\end{frame}

\begin{frame}
    \frametitle{Processos \& Threads}
    \alert{Threads} de um mesmo processo \alert{compartilham}:

    \begin{itemize}
        \item \alert{Espaço de endereçamento}, exceto a pilha, os
            registradores e o contador de programa
        \item Arquivos abertos
        \item $\dots$
    \end{itemize}
\end{frame}

\begin{frame}
    \frametitle{IEEE POSIX Threads}
    \begin{itemize}
        \item Portable Operating System Interface (\alert{POSIX})
        \item Norma internacional IEEE POSIX1 1003.1 C
        \item Apenas a API é normalizada, não a ABI
        \item Threads em nível de usuário, em nível de kernel ou misto
    \end{itemize}
\end{frame}

\begin{frame}
    \frametitle{Bibliotecas}
    \begin{itemize}
        \item LinuxThread (1996)
        \item GNU Pth (1999)
        \item NGPT (2002)
        \item NPTL (2002)
        \item PM2/Marcel (2011)
    \end{itemize}
\end{frame}

\section{Exemplos}

\begin{frame}
    \frametitle{POSIX Threads: Tutorial}
    \alert{POSIX Threads Programming}:
    \begin{itemize}
        \item Blaise Barney, Lawrence Livermore National Laboratory
        \item \url{https://computing.llnl.gov/tutorials/pthreads/}
    \end{itemize}
\end{frame}

\begin{frame}[fragile]
    \frametitle{POSIX Threads: Hello, World!}
    \begin{lstlisting}[basicstyle=\ttfamily\scriptsize]
    #include <pthread.h>
    #include <stdio.h>
    #include <stdlib.h>
    #define NUM_THREADS 5
    void *print_hello(void *threadid){
        long tid;
        tid = (long) threadid;
        printf("Hello World! It's me, thread #%ld!\n", tid);
        pthread_exit(NULL);
    };
    int main(int argc, char *argv[]){
        pthread_t threads[NUM_THREADS];
        int error_code;
        long t;
        for(t = 0; t < NUM_THREADS; t++){
            printf("In main: creating thread %ld\n", t);
            error_code = pthread_create(&threads[t], NULL,
                                        print_hello, (void *) t);
            if (error_code){
                printf("ERROR pthread_create(): %d\n", error_code);
                exit(-1);
            };
        };
        pthread_exit(NULL);
    };
    \end{lstlisting}
\end{frame}

\begin{frame}
    \frametitle{POSIX Threads: Mais Exemplos}
    Exemplos em \url{https://github.com/phrb/aula-pthreads}:
    \begin{itemize}
        \item Hello, World!
        \item Argumentos
        \item \textit{Join}
        \item Servidor IRC: \url{https://github.com/phrb/simple-irc-server}
    \end{itemize}
\end{frame}

\maketitle

\end{document}
